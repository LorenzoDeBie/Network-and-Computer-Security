\documentclass[11pt, faculty=ea]{ugent2016-report}

\usepackage[a4paper, total={6in, 10in}]{geometry}
\usepackage[dutch]{babel}
\usepackage[hidelinks]{hyperref}

\usepackage[
    backend=biber,
    style=ieee,
    sorting=none
]{biblatex}
\addbibresource{library.bib}

\title{Network and Computer Security}
\subtitle{Summary}
\academicyear{2020-2021}
\author{Lorenzo De Bie}  

\begin{document}
\maketitle
\tableofcontents

\chapter{Introduction} \label{cha:introduction}
\section{Security in the media} \label{sec:security-in-the-media}
    \begin{itemize}
        \item Security <=> User friendly: work of security personel goes unnoticed when everything is good, but they get blamed when things go wrong.
        \item Users remains a security risk:
            \begin{itemize}
                \item Due to lack of knowledge: \citetitle{Rodriguez2014} \cite{Rodriguez2014}
                \item Due to incompetence
                \item Information can stell be shared non-digitally
            \end{itemize}
        \item Nobody is safe: \citetitle{Eeckhaut2014} \cite{Eeckhaut2014}
        \item Privacy vs Security: sacrificing privacy so data can be used for security.
            \begin{itemize}
                \item \citetitle{Derix2013} \cite{Derix2013}
                \item \citetitle{Follorou2013} \cite{Follorou2013}
            \end{itemize}
        \item Check yourself using \url{https://haveibeenpwned.com/}
        \item Privacy vs Health: tracing apps in times of COVID-19
        \item Journalists aren't alwasy exactly IT experts \rightarrow remain a critic, remain sceptic
        \item Future trends: blockchains
            \begin{itemize}
                \item mainly used for data integrity through \textbf{public ledgers}
                \item Used to log activity.
                    \begin{itemize}
                        \item Detect malicious operations, hackers, foreign surveillance, database modifications
                        \item Equally important as access restrictions
                    \end{itemize}
            \end{itemize}
        \item Future trends: cyber warface
        \begin{itemize}
            \item Nation wide actions to cause damage or disruption. Can include physical impact and/or harm to human persons
            \item Interesting targets: traffic lights, electricity systems, water filtration, power plants
            \item Stuxnet:
                \begin{itemize}
                    \item Worm that targeted Iranian nuclear facilities, damaging centrifuges and other hardware
                    \item Most likely an American-Israeli cyberweapon
                \end{itemize}
            \item Petya: ransomware or state attack?
                \begin{itemize}
                    \item Focused strongly on Ukraine systems
                    \item Made very little money
                    \item Either very bugge, or very damaging by purpose: permanent removal of files, nuclear power plants, ministries, metros and banks offline, possible link with assassination of Maksym Shapoval
                \end{itemize}
            \item Future trends: IoT: \citetitle{Ford2013} \cite{Ford2013}
        \end{itemize} 
    \end{itemize}
% end section Security in the media

\section{Example Incidents} \label{sec:example-incidents}
    \begin{itemize}
        \item Ashley Madison (2015)
        \item DNC email leak (2016)
        \item Mirai (2016)
        \item Twitter hack (2020)
    \end{itemize}
% end section Example Incidents

\section{Why do we need security? Why Information Security?} \label{sec:why-do-we-need-security}
    \begin{itemize}
        \item Counterpart of securing material objects
            \begin{itemize}
                \item Material object have some \textbf{value}
                \item Can be stolen or damaged
                \item Cost for security/protection takes into account value and risk of theft/damage
            \end{itemize}
        \item Risk of threats against information security is \textbf{much} greater
        \item Value of information sometimes hard to assess, best estimated by damage caused. Losses cannot be undone
        \item Threats against information include:
            \begin{itemize}
                \item \textbf{Loss} of information
                \item \textbf{Forged} information
                \item \textbf{Unauthorised release} of information
                \item \textbf{Repudiation} of information
            \end{itemize}
        \item Value of information systems hard to asses. Systems used to enable service \rightarrow damage when service unavailable or unreliable
        \item Threats against information systems include:
            \begin{itemize}
                \item \textbf{Unavailability}/disruption of service
                \item \textbf{Unauthorised acces} to service
                \item Threats against exchanged information
            \end{itemize}
        \item Security measures for information systems:
        \begin{itemize}
            \item \textbf{Information Security}: encryption, virus scanners, firewalls\dots
            \item Carry some cost (installation, maintenance, computation time)
            \item dependent on risk and potential damage
        \end{itemize}
    \end{itemize}

\printbibliography

\end{document}