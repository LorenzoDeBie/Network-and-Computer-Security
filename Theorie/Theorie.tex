\documentclass[11pt,british,faculty=ea,layout=titlefont,underline=false,titleUppercase=true,titleUnderline=true,hidelinks]{ugent2016-report}

\usepackage[a4paper, total={6in, 10in}]{geometry}
\usepackage[hidelinks]{hyperref}

\usepackage{graphicx}
\graphicspath{./images/}
\usepackage{caption}
\usepackage{subcaption}
\usepackage{wrapfig}

\usepackage[
	backend=biber,
	style=ieee,
	sorting=none
]{biblatex}
\addbibresource{library.bib}

\title{Network and Computer Security}
\subtitle{Summary}
\academicyear{2020-2021}
\author{Lorenzo De Bie}  

\begin{document}
\maketitle
\tableofcontents

\chapter{Introduction} \label{cha:introduction}
\section{Security in the media} \label{sec:security-in-the-media}
	\begin{itemize}
		\item Security <=> User friendly: work of security personel goes unnoticed when everything is good, but they get blamed when things go wrong.
		\item Users remains a security risk:
			\begin{itemize}
				\item Due to lack of knowledge: \citetitle{Rodriguez2014} \cite{Rodriguez2014}
				\item Due to incompetence
				\item Information can stell be shared non-digitally
			\end{itemize}
		\item Nobody is safe: \citetitle{Eeckhaut2014} \cite{Eeckhaut2014}
		\item Privacy vs Security: sacrificing privacy so data can be used for security.
			\begin{itemize}
				\item \citetitle{Derix2013} \cite{Derix2013}
				\item \citetitle{Follorou2013} \cite{Follorou2013}
			\end{itemize}
		\item Check yourself using \url{https://haveibeenpwned.com/}
		\item Privacy vs Health: tracing apps in times of COVID-19
		\item Journalists aren't alwasy exactly IT experts \rightarrow remain a critic, remain sceptic
		\item Future trends: blockchains
			\begin{itemize}
				\item mainly used for data integrity through \textbf{public ledgers}
				\item Used to log activity.
					\begin{itemize}
						\item Detect malicious operations, hackers, foreign surveillance, database modifications
						\item Equally important as access restrictions
					\end{itemize}
			\end{itemize}
		\item Future trends: cyber warface
		\begin{itemize}
			\item Nation wide actions to cause damage or disruption. Can include physical impact and/or harm to human persons
			\item Interesting targets: traffic lights, electricity systems, water filtration, power plants
			\item Stuxnet:
				\begin{itemize}
					\item Worm that targeted Iranian nuclear facilities, damaging centrifuges and other hardware
					\item Most likely an American-Israeli cyberweapon
				\end{itemize}
			\item Petya: ransomware or state attack?
				\begin{itemize}
					\item Focused strongly on Ukraine systems
					\item Made very little money
					\item Either very bugge, or very damaging by purpose: permanent removal of files, nuclear power plants, ministries, metros and banks offline, possible link with assassination of Maksym Shapoval
				\end{itemize}
			\item Future trends: IoT: \citetitle{Ford2013} \cite{Ford2013}
		\end{itemize} 
	\end{itemize}
% end section Security in the media

\section{Example Incidents} \label{sec:example-incidents}
	\begin{itemize}
		\item Ashley Madison (2015)
		\item DNC email leak (2016)
		\item Mirai (2016)
		\item Twitter hack (2020)
	\end{itemize}
% end section Example Incidents

\section{Why do we need security? Why Information Security?} \label{sec:why-do-we-need-security}
	\begin{itemize}
		\item Counterpart of securing material objects
			\begin{itemize}
				\item Material object have some \textbf{value}
				\item Can be stolen or damaged
				\item Cost for security/protection takes into account value and risk of theft/damage
			\end{itemize}
		\item Risk of threats against information security is \textbf{much} greater
		\item Value of information sometimes hard to assess, best estimated by damage caused. Losses cannot be undone
		\item Threats against information include:
			\begin{itemize}
				\item \textbf{Loss} of information
				\item \textbf{Forged} information
				\item \textbf{Unauthorised release} of information
				\item \textbf{Repudiation} of information
			\end{itemize}
		\item Value of information systems hard to asses. Systems used to enable service \rightarrow damage when service unavailable or unreliable
		\item Threats against information systems include:
			\begin{itemize}
				\item \textbf{Unavailability}/disruption of service
				\item \textbf{Unauthorised acces} to service
				\item Threats against exchanged information
			\end{itemize}
		\item Security measures for information systems:
		\begin{itemize}
			\item \textbf{Information Security}: encryption, virus scanners, firewalls\dots
			\item Carry some cost (installation, maintenance, computation time)
			\item dependent on risk and potential damage
		\end{itemize}
	\end{itemize}
% end sec:why-do-we-need-security

\chapter{Basic Concepts} \label{cha:basic-concepts}
	\section{A security model} \label{sec:a-security⁻model}
		\begin{figure}[h]
			\centering
			\includegraphics[width=0.6\textwidth]{images/network-security-model.png}    
		\end{figure}
	% end sec:a-security⁻model

	\section{Security Goals} \label{sec:security-goals}
		Possible exam questions:
		\begin{itemize}
			\item \textbf{Which security goals does this protocol fullfill?}
			\item \textbf{Which security goals per chapter?}
		\end{itemize} 
		\subsection{Confidentiality} \label{sub:confidentiality}
			\begin{itemize}
				\item Data can only be read by those who are allowed to read the data
				\item Applications:
				\begin{itemize}
					\item Communicating confidential data between branches of a corporation
					\item Passwords
					\item Storage of health data
				\end{itemize}
			\end{itemize}
			\begin{figure}[h]
				\centering
				\begin{subfigure}{.40\textwidth}
					\centering
					\includegraphics[width=\linewidth]{images/data-confidentiality-threat.png}
					\caption{Passive attack by Carol: \textbf{eavesdropping} upon information channel}
					\label{fig:eavesdropping}
				\end{subfigure}
				\begin{subfigure}{.50\textwidth}
					\centering
					\includegraphics[width=\linewidth]{images/data-confidentiality-solution.png}
					\caption{Solution to eavesdropping}
					\label{fig:eavesdropping-solution}
				\end{subfigure}
			\end{figure}
			
			\subsubsection{Traffic-flow confidentiality} \label{subsub:traffic-flow-confidentiality}
				\begin{itemize}
					\item Keeping secret who's communicating with whom
					\item Much harder to achieve than data confidentiality
					\item In \figurename{} \ref{fig:eavesdropping-solution} data confidentiality is OK, traffic-flow confidentiality is NOT OK: Carol can still see that Alice is communicating with Bob
				\end{itemize}
			% end subsub:traffic-flow-confidentiality

			\subsubsection{Confidentiality vs Privacy} \label{subsub:confidentiality-vs-privacy}
				Privacy is having the right to choose what information you give away.
				It is a fundamental right, legally protected since long.
				Not every confidentiality requirement involves privacy: intellectual property in a business requires confidentiality, no privacy.
			% end subsub:confidentiality-vs-privacy
		% end sub:confidentiality

		\subsection{Authentication} \label{sub:authentication}
			Authentication is related to \textbf{identification}: it is the \textit{electronic world} equivalent. \textit{Is the person at the other end of the communication who he claims he is?}

			Guaranteeing teh authenticity of a communication is based on:
			\begin{itemize}
				\item \textbf{Entity} authentication: distinguish each entity from another based on collection of data. Each entity has a unique identity.
				\item \textbf{Attribute} authentication. Attribute = characteristic of an entity. Entities are often authenicated through authentication of some of its attributes. Do the communicating parties exhibit the characteristics they claim to have?
				\item \textbf{Data-origin} authentication: does the data indeed originate from the specified source? Important to evaluate wether data is reliable (\textbf{Data Integrity} see \ref{sub:data-integrity}). Different from entity authentication: \textbf{no interaction with data source}.
			\end{itemize}
			\begin{figure}[h]
				\centering
				\begin{subfigure}{.40\textwidth}
					\centering
					\includegraphics[width=\linewidth]{images/authentication-threat.png}
					\caption{Threat}
					\label{fig:authentication-threat}
				\end{subfigure}
				\begin{subfigure}{.50\textwidth}
					\centering
					\includegraphics[width=\linewidth]{images/authentication-threat-solution.png}
					\caption{Solution}
					\label{fig:authentication-threat-solution}
				\end{subfigure}
			\end{figure}
		% end sub:authentication

		\subsection{Access Control/authorization} \label{sub:access-control-authorization}
			\begin{itemize}
				\item Determines which user may access which resource (data, computation time, etc.)
				\item Requires \textbf{authentication of the entity} requesting access to these resources
				\begin{itemize}
					\item System determines to what extent entity may access those resources
					\item Access rights may \textbf{depend on entity itself or its attributes}
				\end{itemize}
			\end{itemize}

			\subsubsection{Illustration 1: access control in OS} \label{subsub:access-control-in-os}
				\begin{itemize}
					\item Authentication through login and password
					\item Access control determined for this user (entity)
						\begin{itemize}
							\item Full access to own files
							\item Limited acces to some other files
							\item No acces to other files
						\end{itemize}
					\item Access rights different from user to user
				\end{itemize}
			% end subsub:access-control-in-os
			
			\subsubsection{Illustration 2: access control to medical database} \label{subsub:access-control-to-medical-database}
				\begin{itemize}
					\item Different rights for different types of Users
					\item Requires authentication based on specific \textbf{attributes}
					\item Access rights depend on attributes of the user
					\item Access rights different from user type to user type (\textbf{roles})
				\end{itemize}
			% end subsub:access-control-to-medical-database

		% end sub:access-control-authorization

		\subsection{Data integrity} \label{sub:data-integrity}
			\begin{itemize}
				\item Guarantee that sent data and received data are identical
					\begin{itemize}
						\item No tampering with data en route
						\item Nothing was added
						\item Nothing was deleted
						\item Nothing was modified
						\item Nothing was replayed
					\end{itemize}
				\item stronger requirement than data origin authentication: data originates from specified source \textbf{AND} isn't changed on the way
				\item Threats
					\begin{itemize}
						\item Messages can be replayed
						\item Messages can be altered
						\item Cannot be solved with confidentiality (encryption): encrypted messages can also be replayed
					\end{itemize}
			\end{itemize}
			\subsubsection{Solution} \label{subsub:data-integrity-solution}
				\begin{figure}[ht]
					\begin{minipage}{0.4\textwidth}
						A security footer containing a sequence number which can only be generated by the sender. This footer has to be generated based on the whole message to prevent tampering to the message itself.
						No need to encrypt the whole message for data integrity, but the message is not confidential if it isn't encrypted.
					\end{minipage}
					\begin{minipage}{0.58\textwidth}
						\centering
						\includegraphics[width=0.7\textwidth]{images/data-integrity-solution.png}
						\caption{Data integrity solution}
					\end{minipage}
				\end{figure}
			% end subsub:data-integrity-solution
		% end sub:data-integrity

		\subsection{Non-repudiation} \label{sub:non-repudiation}
			\begin{itemize}
				\item Sender can't deny having sent the message. Important for receiver. \textit{Prove order has been placed}
				\item Receiver can't deny having received the message. Important for sender. \textit{Prove invoice has been paid}
				\item Both sides need to communicate and `sign' their messages to guarantee non-repudiation for both sides.
			\end{itemize}
		% end sub:non-repudiation

		\subsection{Availability} \label{sub:availability}
			\begin{itemize}
				\item System/service is accessible and usable for authorised Users
				\item Security context $\leftrightarrow$ System design context
			\end{itemize}
			\subsubsection{Threats} \label{subsub:availability-threats}
				\begin{itemize}
					\item DoS: denial-of-service: target swamped by torrent of messages from attacker
					\item DDos: distributed denial-of-service: target swamped by torrent of messages from multiple (and numerous) senders (botnets).
				\end{itemize}
			% end subsub:availability-threats
		% end sub:availability
	% end sec:security-goals

	\section{Security Threats} \label{sec:security-threats}
		Possible exam questions:
		\begin{itemize} \bfseries
			\item Explain the difference between confidentiality, authentication, acces control/authorization, data integrity, non-repudiation and availability.
			\item Which of the above security goals are realized in the network protocols from Chapter 4?
			\item Why are sequence numbers (or nonces) added to messages? Is it a good idea to use a time stamp for this purpose?
			\item Which counter measurements can be taken against DoS and DDoS attacks?
			\item Give 5 examples of active attacks that can be used to compromise the security of a network protocol.
		\end{itemize}
		\begin{itemize}
			\item \textbf{Passive} attacks
				\begin{itemize}
					\item Eavesdropping
					\item Traffic analysis
				\end{itemize}
			\item \textbf{Active} attacks
				\begin{itemize}
					\item Message insertion/modification
					\item Impersonation/masquerade
					\item Replay
					\item DoS
					\item Hijacking (taking over existing connection, where attacker replaces sender or receiver)
				\end{itemize}
		\end{itemize}
		Hackers first seek a weak point in a network (for example through social engineering), second they wil use passive attacks to gain more information. Lastly they'll use active attacks.

		\subsection{Possible attacks} \label{sub:security-threats-possible-attacks}
			\begin{itemize}
				\item Brute force: Trying all possible keys
				\item Cryptanalysis: using knowledge about structure of algorithm, pairs of plaintext and secure messages in order to recover plaintext message or key itself, or to forge secure message
				\item Side-channel attacks use physical properties or fault injection in order to recover plaintext or key
					\begin{itemize}
						\item \citetitle{Cade2014} \cite{Cade2014}
						\item \citetitle{Anthony2013} \cite{Anthony2013}
					\end{itemize}
			\end{itemize}
		% end sub:security-threats-possible-attacks

		\subsection{Categories of attacks} \label{sub:categories-of-attacks}
		\begin{itemize}
			\item \textbf{Ciphertext only}: only secure message is known to attacker. Hardest one to break.
			\item \textbf{Known plaintext}: one or more pairs obtained with a single key are known to attacker. Easier to break, but still safe.
			\item \textbf{Chosen plaintext}: one or more pairs obtained with a single key, plaintext chosen by attacker. Harder to get, easier to break.
			\item \textbf{Chosen ciphertext}: one or more pairs obtained with a single key, ciphertext chosen by attacker (plaintext can be garbage). Even harder to get, easier to break.
			\item \textbf{Chosen text}: combination of chosen plaintext and chosen ciphertext.
		\end{itemize}
		% end sub:categories-of-attacks

		\subsection{Desired degree of security?} \label{sub:desired-degree-of-security}
			\begin{itemize}
				\item Unconditionally secure is \textbf{not achieved by any practical security mechanism}.
				\item Computationally secure means that the \textbf{time required for breaking is longer than the usefull lifetime} of the information, or that the \textbf{cost of breaking the encryption is larger than the value} of the information.
			\end{itemize}
		% end sub:desired-degree-of-security
	% end sec:security-threats
% end cha:basic-concepts

\printbibliography

\end{document}